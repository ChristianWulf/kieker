This chapter gives a brief description on how to use the \class{AMQPWriter} and \class{AMQPReader} %
classes, which allow to use Kieker with AMQP-based queue implementations such as %
RabbitMQ.\footnote{\url{http://www.rabbitmq.com}} The directory \dir{\AMQPBookstoreApplicationReleaseDirDistro/} %
contains the sources, gradle scripts etc.\ used in this example. It is based on the Bookstore %
application with manual instrumentation presented in Chapter~\ref{chap:example}. %

The following paragraphs provide step-by-step instructions for the popular AMQP implementation RabbitMQ.%

\section{Preparation}

\subsection{Download and Install RabbitMQ}
Download the RabbitMQ distribution from \url{http://www.rabbitmq.com/download.html} and follow the installation %
instructions for your OS. Since RabbitMQ requires Erlang, additional software packages may have to be installed %
on your machine.
\par In order to use RabbitMQ's integrated management UI, you may have to enable the appropriate plugin first. This is %
done by issuing the following command from the command line. 

\setBashListing
\begin{lstlisting}[caption=Enable the management UI under UNIX-like systems]
#\lstshellprompt{}# rabbitmq-plugins enable rabbitmq_management
\end{lstlisting}
\begin{lstlisting}[caption=Enable the management UI under Windows]
#\lstshellprompt{}# rabbitmq-plugins enable rabbitmq_management
\end{lstlisting}

Once the UI is enabled, you may access it at port 15672 by default. %

\subsection{Configure RabbitMQ}
Once the RabbitMQ server is installed and started, create a queue for Kieker to use. This can be done easily using %
RabbitMQ's management UI. It is accessible via \url{http://localhost:15672} (the default credentials are \texttt{guest:guest}) We will assume a queue named \parameterValue{kieker} for the remainder of this %
example. Please note the following caveats when configuring the server:

\begin{compactenum}
 \item If you choose to create a transient queue, the entire queue (not just the queued messages) is destroyed %
 on server shutdown and must be re-created manually. %
 \item The RabbitMQ server's default permissions grant access only from \hostname{localhost}. If your RabbitMQ server runs %
 on a remote machine, you have to set the permissions accordingly. %
\end{compactenum}

\subsection{Kieker Monitoring Configuration for RabbitMQ}
The file \file{src-resources/META-INF/kicker.\-monitoring.\-pro\-perties} %
is already configured to use the \class{AMQPWriter}. The important properties are %
the server URI and the queue name. %

\setPropertiesListing
\lstinputlisting[firstline=9,lastline=9,caption=Excerpt from \file{kicker.monitoring.properties} configuring the URI of the AMQP server]{\AMQPBookstoreApplicationDir/src-resources/META-INF/kicker.monitoring.properties}

\setPropertiesListing
\lstinputlisting[firstline=15,lastline=15,caption=Excerpt from \file{kicker.monitoring.properties} configuring the AMQP queue name]{\AMQPBookstoreApplicationDir/src-resources/META-INF/kicker.monitoring.properties}

\section{Running the Example}
The execution of the example is performed by the following three steps:
\begin{enumerate}
\item Ensure that the RabbitMQ server is started and the configured queue is accessible.

\item Start the analysis part (in a new terminal):
\setBashListing
\begin{lstlisting}[caption=Start the analysis part under UNIX-like systems]
#\lstshellprompt{}# ./gradlew runAnalysisAMQP
\end{lstlisting}
\begin{lstlisting}[caption=Start the analysis part under Windows]
#\lstshellprompt{}# gradlew.bat runAnalysisAMQP
\end{lstlisting}
\item Start the instrumented Bookstore (in a new terminal):
\setBashListing
\begin{lstlisting}[caption=Start the analysis part under UNIX-like systems]
#\lstshellprompt{}# ./gradlew runMonitoringAMQP
\end{lstlisting}
\begin{lstlisting}[caption=Start the analysis part under Windows]
#\lstshellprompt{}# gradlew.bat runMonitoringAMQP
\end{lstlisting}
\end{enumerate}